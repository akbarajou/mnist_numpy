\documentclass{article}
\usepackage[utf8]{inputenc}
\usepackage{amsmath, amssymb}
\usepackage{graphicx}

\title{Neural Network from scratch with NumPy for MNIST}
\author{
    Akbar Tangirov\\
    Department of Electrical and Computer Engineering\\
    Ajou University in Tashkent\\
    \texttt{202290398@ajou.uz}
}

\date{\today}

\begin{document}

\maketitle

\begin{abstract}
In this report, we describe the development of a neural network built from the ground up, using only NumPy to train and test on the MNIST dataset. This approach demonstrates a fundamental understanding of deep learning concepts without relying on high-level machine learning frameworks.
\end{abstract}

\section{Introduction}
The MNIST dataset is a benchmark dataset consisting of $28 \times 28$ pixel handwritten digits. Our project aims to implement and understand each component of a neural network, including forward propagation, loss calculation, backpropagation, and parameter updates. 

\section{Methodology}
\subsection{Data Preparation}
We load the MNIST data, normalize it, and reshape images for input into the network. 

\subsection{Network Architecture}
Our network includes an input layer, one or more hidden layers, and an output layer. We use common activation functions such as ReLU or Sigmoid, depending on requirements.

\subsection{Training Process}
We apply mean squared error or cross-entropy loss, then compute gradients during backpropagation. Model parameters are updated using gradient descent.

\section{Results}
After training, the model’s accuracy on the MNIST test set is evaluated. We present a summary of key metrics, confusion matrices, and example predictions.

\section{Conclusion}
Building a neural network from scratch with NumPy deepens understanding of matrix operations, gradient-based optimizations, and the fundamentals of deep learning.

\end{document}